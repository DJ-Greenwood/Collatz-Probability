\section{Introduction}

The Collatz conjecture, also known as the 3n+1 problem, concerns the behavior of sequences generated by iterating a simple function on positive integers. For any positive integer \( n \), the next term in the sequence is given by:

\[
T(n) = 
\begin{cases} 
\frac{n}{2} & \text{if } n \text{ is even} \\ 
3n + 1 & \text{if } n \text{ is odd}
\end{cases}
\]

The conjecture states that, starting from any positive integer \( n \), repeated application of this function will eventually reach the number 1. Despite its simple definition, the conjecture remains unproven and is one of the most famous unsolved problems in mathematics.

This paper presents a probabilistic framework to analyze the Collatz conjecture, focusing on the expected convergence time and the maximum values reached in the sequences. The framework builds on the observation that the behavior of even and odd numbers can be modeled probabilistically, allowing for a deeper understanding of the dynamics of the sequences. The main contributions of this work include:
\begin{itemize}
    \item A probabilistic model that explains the observed behavior of Collatz sequences without requiring strict bounds.
    \item Calculable estimates for the expected convergence time and maximum values reached in the sequences.
    \item Insights into the structural properties of integers under the Collatz.
\end{itemize}
