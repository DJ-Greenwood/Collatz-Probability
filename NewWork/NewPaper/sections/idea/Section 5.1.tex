\section{A Probabilistic Proof by Contradiction for the Collatz Conjecture}

\subsection{Statement of the Collatz Conjecture}
The Collatz conjecture states that for any positive integer \( n \), the sequence defined by the transformation:
\[
T(n) = 
\begin{cases} 
\frac{n}{2}, & \text{if } n \text{ is even} \\
3n + 1, & \text{if } n \text{ is odd}
\end{cases}
\]
eventually reaches the cycle \(\{4, 2, 1\}\).

We aim to prove this by contradiction, assuming that there exists at least one positive integer \( n \) that does not lead to \(\{4, 2, 1\}\) and showing that this assumption contradicts the probabilistic properties derived in the paper.

\subsection{Assumption for Contradiction}
Assume that there exists an initial integer \( n_0 \) whose Collatz sequence does not reach \(\{4, 2, 1\}\). This assumption implies one of two possibilities:
\begin{itemize}
    \item \textbf{Infinite Growth Hypothesis:} The sequence increases indefinitely without bound.
    \item \textbf{New Cycle Hypothesis:} The sequence enters a new, previously undiscovered cycle different from \(\{4, 2, 1\}\).
\end{itemize}
We analyze each case separately.

\subsection{Rejection of the Infinite Growth Hypothesis}
We consider the probability of a sequence increasing indefinitely.

\subsubsection{Transition Probabilities Favor Reduction}
From the paper, we note that:
\begin{itemize}
    \item An odd number always transitions to an even number with probability 1 via the \(3n + 1\) operation.
    \item An even number remains even with probability \(\frac{3}{4}\), ensuring further halving.
\end{itemize}

The expected reduction per step for even numbers is given by:
\[
E[\text{reduction}] = \frac{3}{4} \times \frac{1}{2} + \frac{1}{4} \times \frac{1}{4} = \frac{7}{16}
\]
indicating that numbers generally decrease in value.

The probability of \( k \) consecutive increases (i.e., avoiding a halving step) is at most:
\[
P(k \text{ increases}) \leq \left(\frac{1}{4}\right)^k
\]
which decays exponentially.

\subsubsection{Statistical Impossibility of Infinite Growth}
For \( n \) to grow indefinitely, it must avoid the power-of-2 funnel forever. However:
\begin{itemize}
    \item Each odd step introduces an even number, forcing the sequence into the probabilistic halving process.
\end{itemize}

Since the probability of avoiding halving decreases exponentially, the probability of infinite growth is:
\[
\lim_{k \to \infty} P(\text{infinite growth}) = 0.
\]
Contradiction: Since infinite growth has probability zero, it cannot occur.

Thus, our assumption that a sequence can grow indefinitely is false.

\subsection{Rejection of the New Cycle Hypothesis}
Assume there exists a new cycle \( C \) containing numbers that do not reach \(\{4, 2, 1\}\).

\subsubsection{Structure of a Cycle in the Collatz Process}
For a cycle to exist, every number in the cycle must return to itself after a finite number of steps. However:
\begin{itemize}
    \item The transformation \(3n + 1\) increases odd numbers unpredictably, making exact cycle formation difficult.
    \item The transition probabilities favor reduction, meaning numbers are statistically biased toward lower values.
\end{itemize}

\subsubsection{Expected Value Contraction Contradicts Cyclic Behavior}
The expected transformation of \( n \) can be analyzed probabilistically:
\begin{itemize}
    \item Odd steps introduce even factors, ensuring probabilistic descent.
    \item Even steps halve numbers with high probability, reducing magnitude over time.
\end{itemize}

The expected reduction over multiple steps follows:
\[
E[n_{t+1}] \approx \frac{7}{16} E[n_t]
\]
which implies an overall decreasing trend.

For a cycle to exist, the sequence must return to its original value after a finite number of steps, meaning the expectation must balance between increases and decreases. However, since the process statistically trends downward, this balance cannot hold.

Thus, no new cycle can exist, leading to another contradiction.

\subsection{Conclusion}
Since both assumptions—infinite growth and a new cycle—lead to contradictions, we conclude that:
\begin{itemize}
    \item Every sequence must eventually enter the power-of-2 funnel.
    \item Once within the funnel, the sequence deterministically reduces to \(\{4, 2, 1\}\).
\end{itemize}

Therefore, the Collatz conjecture holds for all positive integers.

Q.E.D.